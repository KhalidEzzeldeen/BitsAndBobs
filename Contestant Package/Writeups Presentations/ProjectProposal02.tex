
\documentclass[letterpaper,12pt,onecolumn]{article}
%\documentclass[12pt]{}
\usepackage{times}
%\input{psfig.sty}
\usepackage{graphicx}
\usepackage{setspace}
\usepackage{amsthm, amssymb}
\usepackage[cmex10]{amsmath}
\usepackage{amsmath}
\usepackage{rotating}

%\usepackage{natbib}
%\usepackage{multirow}
% \bibpunct[, ]{(}{)}{,}{a}{}{,}%
% \def\bibfont{\small}%
% \def\bibsep{\smallskipamount}%
% \def\bibhang{24pt}%
% \def\newblock{\ }%
% \def\BIBand{and}%

%% Setup of theorem styles. Outcomment only one.
%% Preferred default is the first option.
%\TheoremsNumberedThrough     % Preferred (Theorem 1, Lemma 1, Theorem 2)
%\TheoremsNumberedByChapter  % (Theorem 1.1, Lema 1.1, Theorem 1.2)
%\ECRepeatTheorems
%% Setup of the equation numbering system. Outcomment only one.
%% Preferred default is the first option.
%\EquationsNumberedThrough    % Default: (1), (2), ...


\setlength{\textheight}{9.0in} \setlength{\textwidth}{6.5in}
\setlength{\columnsep}{0.3in} \setlength{\topmargin}{-0.4625in}
\setlength{\oddsidemargin}{0.0in} \setlength{\evensidemargin}{0.0in}

%\setlength{\textheight}{9.0in} \setlength{\textwidth}{6.5in}
%\setlength{\columnsep}{0.3in}
%\setlength{\topmargin}{-0.5in}
%\setlength{\oddsidemargin}{-0.2in}
%\setlength{\evensidemargin}{-0.2in} %\setlength{\voffset}{0.0in}


%\def\QED{\mbox{\rule[0pt]{1.5ex}{1.5ex}}}


%\documentclass[12pt]{article}

%\usepackage{graphicx}
%\usepackage{amsmath}


\vspace{96pt}
\title{A Neural Network model to analyse how intervention of BMI would affect incidence of diabetes }
\vspace{96pt}
\author{Thomas Richardson, Yuan Zhang \\
        Operations Research\\
        North Carolina State University\\
        }
\vspace{24pt}
\date{\today}


\begin{document}
\maketitle
\newpage


\section*{Introduction}
There are about 23.6 million children and adults in the United States� 7.8\% of the population� living with  diabetes, and 1.6 million new cases of diabetes are diagnosed in people aged 20 years and older each year \cite{Diabetes2007}. Complications of diabetes including heart disease, stroke, blindness, kidney disease and etc. Overall, the risk for death among people with diabetes is about twice that of people without diabetes of similar age. \cite{Diabetes2007}.

U.S. is spending \$174 billion for patients with diabetes, combining \$116 billion for direct medical costs and \$58 billion for indirect costs (disability, work loss, premature mortality) \cite{Diabetes2007}. Today diabetes prevention and early treatment are under-emphasized, if adults received their recommended diabetes screenings and early lifestyle intervention of  medicine treatment, complications and disability could be avoided and billions of dollars can be saved \cite{Chang2007}. 

SAS has created a contest in which the United States Health Expense Think Tank (USHETH), a fictional group, wants to know the impact of preventive measures on diabetes.  Furthermore, it would like to know how an increase or decrease in diabetes prevalence would affect the amount of money spent on health care. Specifically, USHETH wants to research on the following two questions:

\begin{list}{*}{}
	\item USHETH wants to know diabetes influences a patient total health care expense.  There are many estimates of the cost of diabetes care that have focused on diabetes-specific expenses, but it is known that diabetes might affect total health care expenses in many ways.  There are many factors in the patient database that need to be taken into account so we propose to build a model to know the difference in health care costs for a person with diabetes from the costs for a person without the disease.  It acknowledges that there are many factors that need to be taken into account.

	\item USHETH needs to know how many people have diabetes by age and perhaps gender.  This will be critical for the analysis.
We know that BMI is established as one of the  significant risk factors for diabetes \cite{Weinstein2004}, and studies show that lifestyle intervention can reduce the incidence of diabetes in persons at high risk \cite{Knowler2002}. USHETH wants to measure the impact of diabetes preventive measure. The measure focuses on people who have a BMI (body mass index) larger than 25 in adults.  In children the threshold BMI varies by age linearly: at 5 years old, the threshold BMI is 17 and at 20 years old, the threshold BMI is the same for adults.  These people will be enrolled in special programs to reduce their BMI by ten percent. USHETH then wants to know how many people would contract diabetes.

\end{list}



%position your project within the literature.

In order to complete this analysis, SAS has provided a 50,000+ patient data set with 42 parameters for each patient, as listed in Table \ref{table: parameters}.  We classify the parameters provided into four categories: demographics, general physical characteristics, current diseases and health care characteristics.  The sample provided is representative of the population and provides a snapshot at a point in time.

\begin{table}[htbp]
\caption{\textbf{Parameters in The Data Set}}
\begin{tabular}{|l|l|}
\hline
\textbf{DEMOGRAPHICS} & \textbf{CURRENT DISEASES} \\ \hline
 &  \\ \hline
SEX & HIGH BLOOD PRESSURE DIAGNOSIS \\ \hline
CENSUS REGION & CORONARY HRT DISEASE DIAGNOSIS \\ \hline
AGE & ANGINA DIAGNOSIS \\ \hline
MARITAL STATUS & HEART ATTACK DIAGNOSIS \\ \hline
YEARS OF EDUCATION & OTHER HEART DISEASE DIAGNOSIS \\ \hline
EVER SERVED IN ARMED FORCES & STROKE DIAGNOSIS \\ \hline
DID ANYONE PURCHASE FOOD STAMPS & JOINT PAIN LAST 12 MNTH \\ \hline
TOTAL INCOME & ASTHMA DIAGNOSIS \\ \hline
HAS MORE THAN ONE JOB & DIABETES\_DIAG\_BINARY \\ \hline
 &  \\ \hline
\textbf{GENERAL PHYSICAL CHARACTERISTICS} & \textbf{HEATH CARE CHARACTERISTICS} \\ \hline
  &  \\ \hline
WEARS EYEGLASSES & DENTAL CHECK-UP \\ \hline
PERSON IS BLIND & HOW LONG CHOLEST LAST CHECK \\ \hline
PERSON WEARS HEARING AID & HOW LONG LAST ROUTNE CHECKUP \\ \hline
PERSON IS DEAF & HOW LONG LAST FLU SHOT \\ \hline
PERSON WEIGHT & NUM OFFICE-BASED PROVIDER VISITS \\ \hline
LOST ALL UPPR AND LOWR TEETH & HOW LONG SINCE LAST PSA \\ \hline
ADULT BMI & HOW LONG LAST PAP SMEAR TEST \\ \hline
CHILD BMI & HOW LONG SNCE LAST BREAST EXAM \\ \hline
CURRENTLY SMOKE & HOW LONG SNCE LAST MAMMOGRAM \\ \hline
 & BLOOD STOOL TEST \\ \hline
 & SIGMOIDOSCOPY/COLONOSCOPY \\ \hline
 & WEARS SEAT BELT \\ \hline
\end{tabular}
\label{table: parameters}
\end{table}


In determining how effective the proposed measure will be, we must analyse the onset of diabetes in the stated BMI categories, and determine relevant correlations. Notice that the data provides only a snapshot, and not a time series of data points. Therefore we have little knowledge of how these parameters changed for individuals over time, though we can extrapolate the onset of particular things (diabetes, heart conditions, BMI changes) by trends in the data. Using statistics methods, we can calculate the proportion of diabetes patients in a specified population (in demographics, physical characteristics, and etc). How would the proportion of diabetes change if the specified population reduce BMI by 10\%? 

In order to answer the proposed questions, we have decided to construct a neural network to parse out the relationship between population characteristics (input) and the binary parameter diabetes diagnosis (output).  Many neural network structures have been used in medical analysis \cite{Ripley1998} \cite{Lisboa2006} \cite{Lisboa1994}.  Specifically, Park, et al. \cite{Park2001} built a neural network to evaluate the Heath Risk Appraisal data for diabetes prediction.  They claimed that the use of neural network can enhance the identification of individuals who are at high risk for specific diseases in a time-sensitive manner.  We propose to identify the high risk population for diabetes, then measure the reduction of diabetes risk by reducing BMI by a certain level for the high risk population.  In addition, we propose to examine the threshold of BMI, currently set as 25, that is used to identify the high risk population.


%For each reference state: who, what, how and the most significant findings.

%Explain how your proposal differs from existing literature.

\section*{Model} \label{section: model}

%Provide a rigorous mathematical description of your model carefully define all notation.\newline


%For the first problem, we need to establish some function to describe health care costs for individuals as a function of diabetes, that is, we need to find a relationship between the various parameters provided and health care.  As incidence of diabetes is associated with some other factors in the data set, for example high blood pressure diagnosis, heart disease diagnosis, stroke diagonis, and etc. We propose to 
%  It may also be useful to find relations for total costs, Medicare costs, and Medicaid costs independently.  We propose to use fuzzy clustering and fuzzy regression for this goal.

%The second problem, finding diabetes distributions as a function of age and gender is pretty straightforward.  For this, we just need to find the relevant percentages falling into each category.  From this, information about a critical onset age may revel itself, or a gender bias for the disease.
%
%The second problem will likely be the most difficult, as it is rather open ended beyond the statement of the problem. 

 

We propose to create a three layers feedforward neural network which take the binary parameter of diabetes diagnosis as the output $ Y $ and selected parameters $ (X_{BMI}, X) $ in Table \ref{table: parameters} as input. 

We propose to use the Neural Network Toolbox of Matlab to build, train and validate the model.




%state and justify your assumptions.\newline





\section*{Methodology}



The basic procedure of our analysis is outlined below:
\begin{description}
\item[Step 1:] We propose to filter the data set by eliminating incomplete data. The target data set is called $ A $.
\item[Step 2:]  Construct neural network(NN), $ N1 $, and train via $ A $ or some random sub-population from $ A $.  
The main point is that the proportion of non-diabetes to diabetes patients is equal in the sub-population to the ratio in the data-set. 
\item[Step 3:] Validate $ N1 $ with $ A $; select some sub-set of $ A $ such that $ Y1 $ (the output of $ N1 $ from the input $ A $) is between some range around 0.5. This data set is $ A2 $.

%Step 4: Find some A1' via GA such that the score function selects chromosomes based on their ability to select a subpopulation of A that sits "on the cusp" of having and not having diabetes given the other parameters in the data set (besides health care cost?). Note:  the intersection of A1 and A1' should approximate the union of A1 and A1' (ideally :)

%Step 5: Find A2 = intersection of A1 and A1'.  Construct and train NN, N2, with A2. 

\item[Step 4:] Construct and train NN,  $ N2 $, with $ A2 $. Validate $ N2 $ with $ A2 $; record output $ Y2 $, which is a decimal value between 0 and 1.
\item[Step 5:] Note:  let $ Y_{21} $ be output of $ N2 $ from $ A_{21} $, which is a subset of $ A2 $ with diabetes, and $ Y_{20} $ be output of $ N2  $ from $ A_{20} $, which is the complimentary  of $ A_{21} $.  Examine if there is a statistical difference between $ Y_{21} $ and $ Y_{20} $. Construct a function $f(Y_{21} , Y_{20}) $ to measure the statistical difference.
\item[Step 6:] Construct $ A3 = \{(X_{BMI}*(1-p), X, Y) | (X_{BMI},X,Y) in A2, and X_{BMI} >= Tar_{BMI}\}. $,  given policy $ P = \{p,Tar_{BMI}\} $. Our first test case is $ p=10\%, Tar_{BMI}=25 $.

%Step 5: Construct $ A3 = {(max[X_BMI*(1-p)*(X_BMI >= Tar_BMI), X_BMI*(X_BMI < Tar_BMI)], X, Y) | (X_BMI,X,Y) in A2}. $, where $ p=10\%, Tar_BMI=25 $.
\item[Step 7:] Validate $ N2 $ with $ A3 $; record output $ Y3 $.
\item[Step 8:] Calculate $ diff_Y = f(Y2,Y3) $.  $ f(Y2,Y3)$ gives some indication as to how effective the reduction of BMI at reducing the occurrence of diabetes in an individual, given policy $ P = \{p,Tar_{BMI}\} $. 
\item[Step 9:] Run different policies to examine if $ Tar_{BMI}=25 $ is a good policy value. 

\end{description}

    


%state and motivate the methodological plan.\newline
%use citations to draw from the existing literature.\newline
%provide a plan:
%methodological steps to answer your research question.
%state possible risks. what will you do if your ideas don't work?\newline


%General clustering methods to see how things relate / best indicators


\section*{Results}


\section*{Performance Analysis}


\section*{Conclusion and Discussion}



\bibliographystyle{plain}	% (uses file "plain.bst")
\bibliography{library}		% expects file "myref.bib"

\end{document}\end{Large} 