% Use only LaTeX2e, calling the article.cls class and 12-point type.

\documentclass[12pt, notitlepage, letterpaper]{article}
\usepackage{bm,url,graphicx, amsfonts, amssymb} %PJM added these here
\usepackage{amsmath}    % need for subequations
\usepackage{graphicx}   % need for figures
\usepackage{subfigure}  % use for side-by-side figures


% The following parameters seem to provide a reasonable page setup.

% Include your paper's title here

\title{MA731: Robust HW}

\author{Thomas Richardson}
{\large }

% Include the date command, but leave its argument blank.

% \date{\small}



%%%%%%%%%%%%%%%%% END OF PREAMBLE %%%%%%%%%%%%%%%%



\begin{document}

% Double-space the manuscript.

\baselineskip 24pt

% Make the title.

\maketitle

\noindent

\paragraph{}

Our system is as follows:

%Equations for system, preliminaries
\begin{equation}
\label{eq:sys01}
\dot{x} = Ax + Bu + Dw
\end{equation}
\begin{equation}
\label{eq:sys02}
y = Cx + Ew
\end{equation}
\begin{equation}
\label{eq:sys03}
z = Hx + Gu
\end{equation}

with
\begin{align}
\label{eq:ABSQRBC}
\text{A} 
&=
\begin{bmatrix}
1 & 2 \\
2 & -3
\end{bmatrix}
,&
\text{B}
&=
\begin{bmatrix}
1 \\
0
\end{bmatrix}
,&
\text{C}
&=
\begin{bmatrix}
3 &  1 
\end{bmatrix}
,&
\text{E}
&=
\begin{bmatrix}
0 &  1 
\end{bmatrix}
\end{align}


\begin{align}
\text{D} 
&=
\begin{bmatrix}
0 & 0 \\
1 & 0
\end{bmatrix}
,&
\text{H}
&=
\begin{bmatrix}
0 &  1 
\end{bmatrix}
,&
\text{G}
&=
\begin{bmatrix}
4
\end{bmatrix}.
\end{align}

We also define an uncertainty term $\omega = (w,x(0))$ to be measured by

\begin{equation}
\label{eq:omgNorm}
\|\omega\|^2 = \|x(0)\|^2_X + \int_{0}^{3}w(t)^Tw(t) dt
\end{equation}

Also let

\begin{equation}
\label{eq:etaNorm}
\|\eta\|^2 = \|x(3)\|^2_Y + \int_{0}^{3}z(t)^Tz(t) dt
\end{equation}

In the above equations, $\|x(0)\|^2_X = x(0)^TXx(0)$ and $\|x(3)\|^2_Y = x(3)^TYx(3)$, where $X$ and $Y$ are both the identity matrix of the appropriate size.  

Our stated goal is to obtain values $\gamma$ that are both greater than and less than $\hat{\kappa}$, where $\hat{\kappa}$ is the infinum of values $\kappa_{\mu}$ satisfying $\|\eta\| \leq \kappa_{\mu}\|\omega\|$.  From the source provided, we see that if values of $\kappa_{\mu}$ satisfy the above inequality, then the Riccati equations, $\it{Eq.}$ \ref{eq:RicSig} and $\it{Eq.}$ \ref{eq:RicP}, have solutions $\Sigma(t)$ and $P(t)$, respectively, on the given time interval for the given $\kappa_{\mu}$ value such that $\forall{t} \in[0,T]$,  $\rho(\Sigma(t)P(t)) < \kappa_{\mu}^2$, where $\rho(Mat)$ is the maximum magnitude eigenvalue of $Mat$.


\begin{equation}
\label{eq:RicSig}
\dot{\Sigma} + A \Sigma  + \Sigma A^T - ( \Sigma C^T + L^T)N^{-1} (C \Sigma + L) + \gamma^{-2} \Sigma  Q \Sigma  + M = 0
\end{equation}

\begin{equation}
\Sigma (0) = Y^{-1}
\end{equation}

\begin{equation}
\label{eq:RicP}
\dot{P} + PA + A^TP - (PB + S)R^{-1}(B^TP+S^T) + \gamma^{-2}PMP + Q = 0
\end{equation}

\begin{equation}
P(T) = X
\end{equation}

where

\begin{align}
\begin{bmatrix}
H^tH & H^TG \\
G^TH & G^TG
\end{bmatrix}
&=
\begin{bmatrix}
Q & S \\
S^T & R
\end{bmatrix}
\end{align}

and

\begin{align}
\begin{bmatrix}
DD^T & DE^T \\
ED^T & EE^T
\end{bmatrix}
&=
\begin{bmatrix}
M & L^T \\
L & N
\end{bmatrix}.
\end{align}

Since Assumption A ($R > 0$, $R^{-1}$ bounded, $N > 0$, $N^{-1}$ bounded) and Assumption B (the pairs $(A, D)$ and $(A^T,H^T)$ are stabilizable) hold, then Theorem 3 may be utilized.  Thus, the ARE forms of $\it{Eq.}$ \ref{eq:RicP} and $\it{Eq.}$  \ref{eq:RicSig} were solved and then the condition $\rho(\Sigma^*P^*) < \kappa_{\mu}^2$ was checked, where $\Sigma^*$ and $P^*$ were the ARE solutions to the respective Riccati equations.  The Theorem states that in addition to this, there exists some critical value of $\kappa_{\mu}$ such that numbers below $\kappa_{\mu}$ do not satisfy the conditions stated above, while numbers above $\kappa_{\mu}$ do.  This critical value corresponds to the sought after value, $\hat{\kappa}$.  Thus, using the statements in Theorem 3, the AREs were solved for various $\kappa$ values and the corresponding $P^*$ and $\Sigma^*$ matrices were used to check the required conditions.  In doing so, it was found that the critical value, $\hat{\kappa}$, was somewhere between 5.4 and 5.2.  Additionally, it was found that sufficiently small $\kappa$ values resulted in unsolvable Riccati equations due to singularities.  Thus, $\kappa$ values such as 4 and 2 resulted in solvable Riccati equations that did not meet the above requirements, while $\kappa$ values such as 7 and 20 resulted in solvable AREs with corresponding $\Sigma^*$ and $P^*$ matrices that did satisfy the required conditions.


\end{document}